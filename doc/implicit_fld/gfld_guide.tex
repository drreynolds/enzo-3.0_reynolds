\documentclass[letterpaper,10pt]{article}
\usepackage{geometry}   % See geometry.pdf to learn the layout
                        % options.  There are lots.
\usepackage[latin1]{inputenc}
\usepackage{graphicx}
\usepackage{epstopdf}
\usepackage{amsmath,amsfonts,amssymb}

\DeclareGraphicsRule{.tif}{png}{.png}{`convert #1 `dirname #1`/`basename #1 .tif`.png}

\author{Daniel R. Reynolds}
\title{{\tt gFLDProblem}: \\
A FLD-based Radiation and Chemistry Solver for Enzo}

\renewcommand{\(}{\left(}
\renewcommand{\)}{\right)}
\newcommand{\vb}{{\bf v}_b}
\newcommand{\xvec}{{\bf x}}
\newcommand{\Omegabar}{\bar{\Omega}}
\newcommand{\rhob}{\rho_b}
\newcommand{\dt}{\Delta t}
\newcommand{\Eot}{E^{OT}}
\newcommand{\Ef}{E_f}
\newcommand{\sighat}{\hat{\sigma}}
\newcommand{\Fnu}{{\bf F}_{\nu}}
\newcommand{\Pnu}{\overline{\bf P}_{\nu}}
\newcommand{\R}{I\!\!R}
\newcommand{\Rthree}{\R^3}
\newcommand{\eh}{e_h}
\newcommand{\ec}{e_c}
\newcommand{\Edd}{\mathcal F}
\newcommand{\Eddnu}{\Edd_{\nu}}
\newcommand{\mn}{{\tt n}}
\newcommand{\mB}{\mathcal B}
\newcommand{\mC}{{\mathcal C}}
\newcommand{\mL}{{\mathcal L}}
\newcommand{\mD}{{\mathcal D}}
\newcommand{\mDnu}{\mD_{\nu}}
\newcommand{\mCnu}{\mC_{\nu}}
\newcommand{\mLnu}{{\mathcal L}_{\nu}}
\newcommand{\mCe}{\mC_e}
\newcommand{\mLe}{\mL_e}
\newcommand{\mCn}{\mC_{\mn}}
\newcommand{\mLn}{\mL_{\mn}}


\textheight 9truein
\textwidth 6.5truein
\addtolength{\oddsidemargin}{-0.25in}
\addtolength{\evensidemargin}{-0.25in}
\addtolength{\topmargin}{-0.5in}
\setlength{\parindent}{0em}
\setlength{\parskip}{2ex}


\begin{document}
\maketitle

\section{Introduction}
\label{sec:intro}

This document is meant to describe a new, highly scalable,
field-based radiation and chemistry solver solver for Enzo, 
{\tt gFLProblem}. The target applications of this solver include the
transport of radiation using a flux-limited-diffusion-based solver
over a uniform grid, distributed among a possibly large number of
processors.  In this solver, the radiation field is assumed to be
either a monochromatic radiation energy at the ionization threshold of
Hydrogen I ($h\nu = 13.6$ eV), or an integrated radiation energy
density with an assumed radiation spectrum.  This radiation field may
be coupled to the surrounding matter using either an assumption of
local thermodynamic equilibrium (no chemical ionization), or a model
for Hydrogen ionization (no Helium, yet). 

The defining characteristic of this solver in comparison 
with {\tt gFLDSplit} is that in a given time step, this
solver evolves the radiation, a gas energy correction, and possibly a
Hydrogen-I number density together, using a full nonlinearly-implicit
solver.  The goal in using a nonlinear solver on this problem is
to self-consistently capture the stiff interactions between these
processes in a time-accurate and stable fashion.  In addition to the
full nonlinear solver, this module allows increased choice over
modeling parameters, as it is used as a test-bed for new methods.

This guide will only highlight the solvers and equations available in
this module.  For further details on the equations, numerical methods,
and verification tests relevant to this module, we refer to the paper 
\cite{ReynoldsHayesPaschosNorman2009}.




\section{Flux-limited diffusion radiation model}
\label{sec:rad_model}

We begin with the equation for flux-limited diffusion radiative
transfer in a cosmological medium \cite{ReynoldsHayesPaschosNorman2009},
\begin{equation}
\label{eq:radiation_PDE}
  \partial_{t} E + \frac1a \nabla\cdot\(E\vb\) =
    \nabla\cdot\(D\,\nabla E\) - \frac{\dot{a}}{a} E - c\kappa E + 4\pi\eta,
\end{equation}
where here the comoving radiation energy density $E$ and emissivity
$\eta$ are functions of space and time.  In this equation, the
frequency-dependence of the radiation energy has been integrated away,
under the premise of an assumed radiation energy spectrum,
\begin{align}
  \notag
  & E_{\nu}(\nu,\xvec,t) = E(\xvec,t) \chi(\nu), \\
  \notag
  \Rightarrow & \\
  \label{eq:spectrum}
  & E(\xvec,t) = \(\int_{\nu_0}^{\infty} E_{\nu}(\nu,\xvec,t)\,\mathrm{d}\nu \)
    \Big/ \( \int_{\nu_0}^{\infty} \chi(\nu)\,\mathrm{d}\nu \).
\end{align}
We note that if the assumed spectrum is the Dirac delta function,
$\chi(\nu) = \delta_{\nu_{HI}}(\nu)$, $E$ is a monochromatic radiation
energy density at the ionization threshold of HI, and the
$-\frac{\dot{a}}{a}E$ term (obtained by integration by parts of the
redshift term $\frac{\dot{a}}{a}\partial_{\nu}E_{\nu}$) is omitted
from \eqref{eq:radiation_PDE}. Similarly, the emissivity function 
$\eta(\xvec,t)$ relates to the true emissivity
$\eta_{\nu}(\nu,\xvec,t)$ as  
\begin{equation}
\label{eq:emissivity}
  \eta(\xvec,t) = \( \int_{\nu_0}^{\infty}\eta_{\nu}(\nu,\xvec,t)\,\mathrm{d}\nu \) 
  \Big/ \( \int_{\nu_0}^{\infty}\chi(\nu)\,\mathrm{d}\nu \).
\end{equation}

The function $D$ in the above equation \eqref{eq:radiation_PDE} is
the {\em flux-limiter} that depends on $E$, $\nabla E$ and the 
opacity $\kappa$ (see
\cite{HayesNorman2003,ReynoldsHayesPaschosNorman2009}),  
\[
   D(E) = \text{diag}\( D_1(E),\, D_2(E),\, D_3(E) \),
\]
where the directional limiters $D_i(E)$ are chosen to be one of 
\begin{align}
  \label{eq:ratLP_limiter}
   D_i(E) &= \frac{c(2\kappa+R_i)}{\omega(6\kappa^2+3\kappa R_i + R_i^2)}, \\
  \label{eq:Reynolds_limiter}
   D_i(E) &= \frac{2 c \tan^{-1}\left(\frac{R_i \pi}{6 \kappa}\right)}{\pi
     \omega R_i},  \\
  \label{eq:no_limiter}
   D_i(E) &= \frac{c}{3\kappa}, \\
  \label{eq:Zeus_limiter}
   D_i(E) &= \frac{c(2\kappa+R_i)}{6\kappa^2+3\kappa R_i+R_i^2}, \\
  \label{eq:LP_limiter}
   D_i(E) &= \frac{c \tanh\left(\frac{R_i}{\kappa}\right)-c \frac{\kappa}{R_i}}{\omega R_i}.
\end{align}
In all of the above, the ``effective albedo'' is given by
\begin{equation}
\label{eq:albedo}
  \omega = \frac{(4 \sigma_{SB} T^4)}{c E},
\end{equation}
and the limiter factor $R_i$ is given by either
\begin{align}
  \label{eq:R_zeus}
   R_i &= \frac{|\partial_i E|}{E}, \qquad\text{[Zeus limiter,
     \eqref{eq:Zeus_limiter}]},\\ 
   R_i &= \frac{|\partial_i E|}{\omega E}, \qquad\text{[all others]}.
\end{align}



\section{LTE couplings}
\label{sec:lte_model}

For problems in which chemical ionization is unimportant, we may
assume that the gas is in local thermodynamic eqiulibrium.  In this
case we do not need to couple the radiation energy to a model for
chemical ionization, and only couple the radiation to the specific gas
energy equation, 
\begin{align}
  \label{eq:cons_energy}
  \partial_t e + \frac1a\vb\cdot\nabla e &=
    - \frac{2\dot{a}}{a}e
    - \frac{1}{a\rhob}\nabla\cdot\left(p\vb\right) 
    - \frac1a\vb\cdot\nabla\phi + G - \Lambda.
\end{align}
The advection, pressure, and gravity-dependent terms in
\eqref{eq:cons_energy} are already handled by Enzo's existing
hydrodynamics solver infrastructure.  In this module, we therefore
consider a specific energy correction equation 
\begin{align}
  \label{eq:cons_energy_correction}
  \partial_t e_c &= -\frac{2\dot{a}}{a}e_c + G - \Lambda,
\end{align}
that we use to correct Enzo's original gas energy to include radiation
couplings.  Here $G$ is the local heating rate,
\begin{align}
\label{eq:G_LTE}
  G &= \frac{c \kappa}{\rhob} E,
\end{align}
and $\Lambda$ corresponds to the local cooling rate,
\begin{align}
\label{eq:Lambda_LTE}
  \Lambda = \frac{4\pi}{\rhob} \eta.
\end{align}
The user-defined energy mean opacity $\kappa$ is
spatially-homogeneous, and is given by the formula
\begin{align}
\label{eq:opacityE}
  \kappa = C_0 \left(\frac{\rhob}{C_1}\right)^{C_2}
    \left(\frac{T}{C_3}\right)^{C_4} 
\end{align}
(the constants $C_0\to C_4$ are input by the user), and $\eta$ is
a black-body emissivity given by 
\begin{align}
\label{eq:etaBB}
  \eta = \frac{\kappa_P\,\sigma_{SB}\,T^4}{\pi},
\end{align}
where the user-defined Planck-mean opacity $\kappa_P$ is defined
similarly to the energy-mean opacity \eqref{eq:opacityE},
$\sigma_{SB}$ is the Stefan-Boltzmann constant [$5.6704\times 10^{-5}$
erg s$^{-1}$ cm$^{-2}$ K$^{-4}$], and $T$ is the gas temperature [K].



\section{Chemistry-dependent couplings}
\label{sec:chem_model}

In general, radiation calculations in Enzo are used in simulations
where chemical ionization states are important.  For these situations, 
we couple the radiation equation \eqref{eq:radiation_PDE} with
equations for both the specific gas energy correction and the
ionization dynamics of Hydrogen,
\begin{align}
  \notag
  \partial_t e_c &= -\frac{2\dot{a}}{a}e_c + G - \Lambda, \\
  \label{eq:hydrogen_ionization}
  \partial_t \mn_{HI} + \frac{1}{a}\nabla\cdot\(\mn_{HI}\vb\) &=
    \alpha^{rec} \mn_e \mn_{HII} - \mn_{HI} \Gamma_{HI}^{ph}. 
\end{align}
Here, $\mn_{HI}$ is the comoving Hydrogen I number density.  The
recombination rate $\alpha^{rec}$ may be chosen as either the case-B
recombination rate, 
\begin{equation}
\label{eq:alphaB}
\alpha^{rec} = 2.753\times 10^{-14} \left(\frac{3.15614\times 10^5}{T}\right)^{3/2} 
                   \left(1+\left(\frac{3.15614\times 10^5}{2.74\, T}\right)^{0.407}\right)^{-2.242} 
\end{equation}
or the case-A rate found in Enzo's chemistry rate lookup tables.

In this case, the gas heating and cooling rates are
chemistry-dependent, 
\begin{align}
  \label{eq:G_nLTE}
  G &= \frac{c\,E\,\mn_{HI}}{\rhob} 
    \left[\int_{\nu_{HI}}^{\infty} \sigma_{HI}\, \chi_E
    \left(1-\frac{\nu_{HI}}{\nu}\right)\, d\nu\right] \bigg/
    \left[\int_{\nu_{HI}}^{\infty} \chi_E d\nu\right], \\
\label{eq:Lambda_nLTE}
  \Lambda &= \frac{\mn_e}{\rhob}\bigg[\text{ce}_{HI}\, \mn_{HI} 
  + \text{ci}_{HI}\, \mn_{HI} + \text{re}_{HII}\, \mn_{HII} + \text{brem}\,
  \mn_{HII} \\
  \notag &\qquad+ \frac{m_h}{\rho_{units}\, a^3} \left(\text{comp}_1\, (T-\text{comp}_2) 
    + \text{comp}_{X}\, (T-\text{comp}_{T})\right) \bigg].
\end{align}
The temperature-dependent cooling rates
$\text{ce}_{HI}$, $\text{ci}_{HI}$, $\text{re}_{HII}$, $\text{brem}$,
$\text{comp}_1$, $\text{comp}_2$, $\text{comp}_{X}$ and
$\text{comp}_{T}$ are all taken from Enzo's built-in rate tables.

Moreover, the frequency-integrated opacity is now chemistry-dependent,
\begin{equation}
\label{eq:opacityHI}
  \kappa \ = \ 
  \left[\int_{\nu_{HI}}^{\infty} \kappa_{\nu}\,E_{\nu}\,d\nu\right] \bigg/
  \left[\int_{\nu_{HI}}^{\infty} E_{\nu}\,d\nu\right] \ = \ 
  \left[\mn_{HI} \int_{\nu_{HI}}^{\infty}
    \chi_E\,\sigma_{HI}\,d\nu\right] \bigg/
  \left[\int_{\nu_{HI}}^{\infty} \chi_E\,d\nu\right],
\end{equation}
where these integrals with the assumed radiation spectrum handle the
change from the original frequency-dependent radiation equation to the
integrated grey radiation equation.




\section{Numerical solution approach}
\label{sec:solution_approach}

We solve these models in an implicit-time fashion, coupling either 
\eqref{eq:radiation_PDE}, \eqref{eq:cons_energy_correction} and
possibly \eqref{eq:hydrogen_ionization} together to form a large
system of nonlinear PDEs that must be solved in each time step to find
the updated solution.  This solve is coupled to Enzo's existing
operator-split solver framework in the following manner:
\begin{itemize}
\item[(i)] Project the dark matter particles onto the finite-volume
  mesh to generate a dark-matter density field $\rho_{dm}$ [Enzo];
\item[(ii)] Solve for the gravitational potential $\phi$ using a
  Poisson equation [Enzo];
\item[(iii)] Advect the dark matter particles with the Particl-Mesh
  method [Enzo];
\item[(iv)] Evolve the hydrodynamics equations using an up to
  second-order explicit method, and have the velocity $\vb$ advect
  both the Hydrogen I number density $\mn_{HI}$ and the grey radiation
  field $E$ [Enzo]; 
\item[(v)] Evolve the radiation, gas energy correction and Hydrogen I
  number density implicitly in time [{\tt gFLDProblem}].
\end{itemize}

The implicit solution approach for step (v) is described in detail in 
\cite{ReynoldsHayesPaschosNorman2009}; here we describe only enough
to point out the available user parameters, and more fully describe
some additional options available in the solver.

\subsection{Time discretizations}
\label{sec:iqss}

We must first discretize the equations \eqref{eq:radiation_PDE},
\eqref{eq:cons_energy_correction} and \eqref{eq:hydrogen_ionization}
in space and time before we solve them computationally.  We use a
method of lines approach for the space-time discretization, wherein we
first discretize in space, and then evolve the resulting system of
ODEs in time.  As with the rest of Enzo, we use a finite-volume
spatial discretization, placing all of our unknowns at the center of
each finite-volume cell, and performing all spatial derivatives
through a divergence of face-centered fluxes.  

However, this module allows two different time discretizations of our
ODE system \eqref{eq:radiation_PDE}, \eqref{eq:cons_energy_correction}
and \eqref{eq:hydrogen_ionization}.  The first, described in
\cite{ReynoldsHayesPaschosNorman2009}, employs a two-level
$\theta$-method for the time discretization, and results in the
equations to compute the time-evolved solution $(E^n,e_c^n,\mn_{HI}^n)$,
\begin{align}
  \label{eq:radiation_PDE_theta}
  E^n - E^{n-1} &- \theta\dt\left(\nabla\cdot\(D\,\nabla E^n\) - \frac{\dot{a}}{a} E^n -
    c\kappa^n E^n + 4\pi\eta^n\right) \\ 
  \notag
  & - (1-\theta)\dt\left(\nabla\cdot\(D\,\nabla E^{n-1}\) - \frac{\dot{a}}{a} E^{n-1} -
    c\kappa^{n-1} E^{n-1} + 4\pi\eta^{n-1}\right) = 0, \\ 
  \label{eq:energy_correction_theta}
  e_c^n - e_c^{n-1} &- \theta\dt\left(-\frac{2\dot{a}}{a}e_c^{n} + G^{n} -
    \Lambda^{n}\right) - (1-\theta)\dt\left(-\frac{2\dot{a}}{a}e_c^{n-1} + G^{n-1} -
    \Lambda^{n-1}\right) = 0, \\
  \label{eq:hydrogen_theta}
  \mn_{HI}^n - \mn_{HI}^{n-1} &-
    \theta\dt\left(\alpha^{rec,n} \mn_e^{n} \mn_{HII}^{n} -
      \mn_{HI}^{n} \Gamma_{HI}^{ph,n}\right) -
    (1-\theta)\dt\left(\alpha^{rec,n-1} \mn_e^{n-1} \mn_{HII}^{n-1} -
      \mn_{HI}^{n-1} \Gamma_{HI}^{ph,n-1}\right) = 0,
\end{align}
where $0\le\theta\le 1$ defines the time-discretization, and where we
have assumed that the advective portions of \eqref{eq:radiation_PDE}
and \eqref{eq:hydrogen_ionization} have already been taken care of
through Enzo's hydrodynamics solver.  Recommended values of $\theta$
are 1 (backwards Euler) and $\frac12$ (trapezoidal,
a.k.a.~Crank-Nicolson).  Benefits of this approach include:
\begin{itemize}
\item Standard, easily understandable approach,
\item Unified treatment of all variables in implicit system,
\item Ease of analytical Jacobians for efficient solution of the
  resulting implicit systems.
\item Asymptotic accuracy of $O(\dt)$ for $\theta=1$ and $O(\dt^2)$
  for $\theta=\frac12$. 
\end{itemize}
However, a key drawback of this approach is that it knows nothing
about the inherent constraints on the solution variables, i.e.~$e>0$
and $\mn_{HI}\ge 0$.  As a result, in problems with rapidly-varying
chemical states in which the $\mn_{HI}$ value in a newly-irradiated
finite volume cell should change by multiple orders of magnitude in a
single time step, these simple linear or quadratic interpolations in
time can result in negative solution values due to time discretization
error.  Therefore, in using the above system
\eqref{eq:radiation_PDE_theta}-\eqref{eq:hydrogen_theta}, it is
imperative that the user allow {\em very conservative} time step sizes
to be used (see section \ref{sec:dt_selection}).

As a result, we have also implemented a second approach to time
discretization that should be much more robust, while attempting to
remain as accurate as possible.  This approach is based on a new
implicit-time version of the {\em quasi-steady-state approximation},
the explicit-time version of which is popular in the computational
chemistry community.  In this approach, instead of approximating the
solution to the correct ODEs, we analytically solve approximate ODEs.
Here, if we assume in each equation that all but the time-evolving
variable are held constant throughout the time step, we could consider
the ODE system
\begin{align}
  \label{eq:energy_correction_qss}
  \partial_t e_c &= -\frac{2\dot{a}}{a}e_c + G(\overline{E},\overline{\mn_{HI}}) - \Lambda(\overline{E},e,\overline{\mn_{HI}}), \\
  \label{eq:hydrogen_qss}
  \partial_t \mn_{HI} &= \alpha^{rec}(\overline{e}) \mn_e \mn_{HII} - \mn_{HI} \Gamma_{HI}^{ph}(\overline{E}),
\end{align}
where $\overline{u}$ denotes a field $u$ that is assumed fixed
throughout a time step.  With this approximation,
\eqref{eq:energy_correction_qss}-\eqref{eq:hydrogen_qss} may be
written as
\begin{align}
  \label{eq:energy_correction_qss2}
  \partial_t e_c &= Q - Pe_c, \\
  \label{eq:hydrogen_qss2}
  \partial_t \mn_{HI} &= a \mn_{HI}^2 + b\mn_{HI} + c,
\end{align}
which may be solved analytically to full accuracy for any time step
size $\dt$.  We write these analytical solvers as
\begin{align}
  \label{eq:energy_correction_qss3}
  e_c(t) &= \text{sol}_e\left(\overline{E},\overline{\mn_{HI}},e_c^{n-1},t\right), \\
  \label{eq:hydrogen_qss3}
  \mn_{HI}(t) &= \text{sol}_{HI}\left(\overline{E},\overline{e},\mn_{HI}^{n-1},t\right).
\end{align}
Since such an approach does not work well for PDEs (e.g.~equation
\eqref{eq:radiation_PDE}), and since the radiation equation is
typically less problematic than the gas energy and chemistry, we still
use the $\theta$-method to discretize $E$.  Putting these together,
our second approach to time discretization uses the equations
\begin{align}
  \label{eq:radiation_PDE_iqss}
  E^n - E^{n-1} &- \theta\dt\left(\nabla\cdot\(D\,\nabla E^n\) - \frac{\dot{a}}{a} E^n -
    c\kappa^n E^n + 4\pi\eta^n\right) \\ 
  \notag
  & - (1-\theta)\dt\left(\nabla\cdot\(D\,\nabla E^{n-1}\) - \frac{\dot{a}}{a} E^{n-1} -
    c\kappa^{n-1} E^{n-1} + 4\pi\eta^{n-1}\right) = 0, \\ 
  \label{eq:energy_correction_iqss}
  e_c^n &- \text{sol}_e\left(\frac{E^{n-1}+E^n}{2},\frac{\mn_{HI}^{n-1}+\mn_{HI}^n}{2},e_c^{n-1},t\right) = 0, \\
  \label{eq:hydrogen_iqss}
  \mn_{HI}^n &- \text{sol}_{HI}\left(\frac{E^{n-1}+E^n}{2},\frac{e^{n-1}+e^n}{2},\mn_{HI}^{n-1},t\right) = 0.
\end{align}
Benefits of this approach (as opposed to the $\theta$ method) include:
\begin{itemize}
\item Robust solvers for gas energy and chemistry that can {\em never}
  result in negative solution values,
\item Fully implicit formulation in which all updated solution values
  $E^n$, $e_c^n$ and $\mn_{HI}^n$ depend on one another,
\item Asymptotic time accuracy of $O(\dt^2)$ at $\theta=\frac12$,
  and remarkably good accuracy at even large $\dt$.
\end{itemize}
Unfortunately, however, the use of these solvers makes analytical
derivation of Jacobians very difficult, so they must be approximated,
requiring additional computation per time step.


\subsection{Inexact Newton-Krylov-Multigrid solver}
\label{sec:nkmg}
No matter which time integration strategy is selected from section
\ref{sec:iqss},
\eqref{eq:radiation_PDE_theta}-\eqref{eq:hydrogen_theta} or 
\eqref{eq:radiation_PDE_iqss}-\eqref{eq:hydrogen_iqss}, we solve a
nonlinear system of equations 
\begin{align}
  \notag
  f_E(E,e_c,\mn_{HI}) &= 0, \\
  \label{eq:nonlinear_system}
  f_e(E,e_c,\mn_{HI}) &= 0, \\
  \notag
  f_{HI}(E,e_c,\mn_{HI}) &= 0,
\end{align}
at each time step to obtain the updated solution variables $E^n$,
$e_c^n$ and $\mn_{HI}^n$.  Grouping these together as $f(U)=0$, for
$U=(E,e_c,\mn_{HI})$, we solve the system \eqref{eq:nonlinear_system}
using a {\em globalized Inexact Newton's Method}
\cite{Kelley1995,KnollKeyes2004,KeyesReynoldsWoodward2006}:
\begin{quote}
Given an initial guess $U_0\approx U(t^n)$, we iterate toward a
solution $U^n$ satisfying $f(U^n)\approx 0$:
\begin{itemize}
\item[(a.)] Approximately solve the linear system $J_k S_k = -f_k$:
\begin{itemize}
  \item We find $S_k$ so that $\|J(U_k) S_k + f(U_k)\| <
    \delta_k$. 
  \item $\delta_k$ is a linear tolerance, chosen as $\delta_k =
    C\|f(U_k)\|$, where $C$ is a user-defined input.
  \item $J(U_k)$ is the Jacobian of $f$, $J(U_k) = \partial_U f(U_k)$
  \item We solve this system using a multigrid-preconditioned
    conjugate gradient iteration.
  \end{itemize}
\item[(b.)] Find a line-search parameter $\lambda_k$ so that
  \[
    \lambda_{min} \le \lambda \le \lambda_{max}
    \qquad\text{and}\qquad \|f(U_k + \lambda_k S_k)\| < \|f(U_k)\|.
  \] 
\item[(c.)] Update $U_{k+1} = U_k + S_k$.
\item[(d.)] Stop the iteration when
  \[
    \left(\frac1N \sum_{i=1}^N f(U_k)^2 \right)^{1/2} \ < \
    \varepsilon \ \ll \ 1.
  \]
\end{itemize}
\end{quote}



\subsection{Solver initial guess}
\label{sec:initial_guess}

Performance of the Newton iteration is highly influenced by a good
initial guess: a good guess results in a robust, and efficient method,
while a poor guess may take much longer to converge (if at all).  We
therefore have a number of options that may be used to determine the
initial guess $U_0$:
\begin{itemize}
\item[0.] Use the previous solution 
  $U_0 = U^{n-1}$.
\item[1.] Use an explicit predictor: $U_0 = U^{n-1} + \dt
  \mathcal L(U^{n-1})$, where $\mathcal L$ comprises all of the
  spatially-local physics (i.e.~no diffusion).
\item[2.] Use an explicit predictor: $U_0 = U^{n-1} + \dt
  \mathcal R(U^{n-1})$, where $\mathcal R$ comprises all of the
  physics.
\item[3.] Use a partial explicit predictor: $U_0 = U^{n-1} + \frac{\dt}{10}
  \mathcal L(U^{n-1})$.
\item[4.] Use a partial explicit predictor: $U_0 = U^{n-1} + \frac{\dt}{10}
  \mathcal R(U^{n-1})$.
\item[5.] Use an analytic predictor of spatially-local physics: $U_0
  = (E^{n-1},\text{sol}_e,\text{sol}_{HI})$. 
\end{itemize}

\subsection{Time-step selection}
\label{sec:dt_selection}

\subsection{Variable rescaling}
\label{sec:variable_rescaling}

\subsection{Boundary conditions}
\label{sec:boundary_conditions}




\section{{\tt gFLDProblem} usage}
\label{sec:module_usage}

In order to use the implicit FLD radiation solver module, and to
allow optimal control over the nonlinear and linear solution methods
used, there are a number of parameters than may be supplied to Enzo.
We group these into two categories, those associated with the general
startup of the module via the Enzo infrastructure, and those that may
be supplied to the module itself.  However, prior to embarking on a
description of these parameters, there are a few requirements for any
problem that wishes to use the {\tt gFLDProblem} module.

Foremost, Enzo must be built using the configuration option 
{\tt RAD\_HYDRO}; this is enabled through the call 
{\tt gmake rad-hydro-yes} in the Enzo source directory.  Moreover, the
machine Makefile must specify how to include and link with an
available HYPRE library (version $\ge$ 2.4.0b).  If a user must
compile HYPRE themselves to obtain this version, they should make note
of the HYPRE configuration {\tt --with-no-global-partition}, which
must be used for solver scalability when using over $\sim1000$
processors, but which results in slower executables on smaller
problems.


\subsection{Startup parameters}

In a user's main problem parameter file, the following three parameters
must be set:
\begin{itemize}
\item {\tt RadiationHydrodynamics} -- this must be set to either 1 or
  2.  A value of 1 enables the use of implicit radiation solvers
  alongside standard Enzo chemistry and hydrodynamics physics.  A
  value of 2 turns on the implicit radiation solvers, and {\em turns off}  
  all other Enzo physics solvers.
\item {\tt ImplicitProblem} -- this must be set to 2 to use the 
  {\tt FSProb} module (among the possible implicit radiation solver
  modules).
\item {\tt RadHydroParamfile} -- this should contain the filename
  (relative to this parameter file) that contains all module-specific
  solver parameters.  While the {\tt FSProb} module parameters may be
  supplied in the main parameter file, it is not recommended since
  Enzo's main {\tt ReadParameterFile.C} routine will complain about
  all of the 'unknown' parameters.
\end{itemize}

In addition, in a user's problem initialization routines, they must
allocate a standard Enzo baryon field having the {\tt FieldType} set
to {\tt RadiationFreq0}.  It is this baryon field that will be evolved
by the {\tt FSProb} module, and that a user may access to obtain
information on the free-streaming radiation field.

Furthermore, the {\tt FSProb} module currently computes its own
emissivity field $\eta_f(\xvec)^n$, in the routine 
{\tt FSProb\_RadiationSource.src90}.  A user may edit this file to add
in an emissivity field corresponding to their own Enzo 
{\tt ProblemType}.  Alternatively, we are currently working to
implement a separate interface whereby some other Enzo function fills
in an emissivity field, which is then used internally by the 
{\tt FSProb} module.

For convenience, we have provided the {\tt ProblemType} 250, with the
initialization filed {\tt FSMultiSourceInitialize.C} and 
{\tt Grid\_FSMultiSourceInitializeGrid.C}, as an example of how to set
up this module.



\subsection{Module parameters}

Once a user has enabled the free-streaming radiation module, they have
complete control over a variety of internal module parameters.  The
parameters are given here, with their default values specified in brackets.
\begin{itemize}
\item {\tt FSRadiationScaling} [1.0] -- this parameter may be used to
  rescale the internal solver units for the free-streaming radiation
  field, to ensure that the linear system is well-scaled to have
  solution with values on the order of 1.  We note that without
  scaling, the solver will use the internal Enzo units for a radiation
  field, i.e.~the radiation field will have values in 
  CGS/({\tt DensityUnits*VelocityUnits*VelocityUnits}). 
\item {\tt FSRadiationTheta} [1.0] -- this is the implicit
  discretization method parameter.  1 gives implicit Euler, and 2
  gives the trapezoid rule (Crank-Nicolson).
\item {\tt FSRadiationOpacity} [$10^{-4}$] -- this is the opacity
  constant used within the computations of the flux-limiter
  $D_f(\Ef)$.  The expected values should be those of a (small)
  physical opacity for the problem in CGS.
\item {\tt FSRadiationLimiterType} [4] -- this gives the formula for
  the flux limiter to be used in the code.  The default value provides
  the formula \eqref{eq:limiter}; other allowable values and their
  formulas may be found in the file {\tt FSProb\_SetupSystem.src90}.
\item {\tt FSRadiationBoundaryX0Faces} [0 0] -- these integer
  constants give the type of boundary conditions to supply on the
  free-streaming radiation field at the lower and upper domain
  boundaries in the x-direction, respectively.  Allowable 
  constants are 0 (periodic), 1 (Dirichlet), and 2 (Neumann).  If
  either 1 or 2 are provided, the file {\tt FSProb\_Initialize} will
  need to be modified to set the Dirichlet or Neumann values on the
  appropriate faces, based on the {\tt ProblemType}.
\item {\tt FSRadiationBoundaryX1Faces} [0 0] -- lower and upper
  boundary condition types for the y-direction.
\item {\tt FSRadiationBoundaryX2Faces} [0 0] -- lower and upper
  boundary condition types for the z-direction.
\item {\tt FSRadiationMaxDt} [0.0] -- this gives a maximum allowed
  time step size for the free-streaming radiation problem.  The
  default value of 0 turns off control over the maximum step size; any
  positive value will turn it on.
\item {\tt FSRadiationInitialGuess} [0] -- this gives control over the
  algorithm to use in computing the initial guess at the time-evolved
  solution.  Since the solver is iterative, a good initial guess will
  result in faster convergence.  Allowable values may be found in the
  file {\tt FSProb\_InitialGuess.src90}.
\item {\tt FSRadiationTolerance} [$10^{-10}$] -- this gives the
  absolute tolerance to be used in solving the linear system
  \eqref{eq:fs_linear}.  Note, this absolute tolerance is used on the
  scaled linear system (see {\tt FSRadiationScaling}).
\item {\tt FSRadiationMaxMGIters} [50] -- this gives the maximum
  number of multigrid v-cycles to be used.
\item {\tt FSRadiationMGRelaxType} [1] -- this gives the type of
  smoother to use within multigrid v-cycles.  Values of 0 or 1 give
  weighted Jacobi, and values of 2 or 3 give red-black Gauss-Seidel
  (see HYPRE documentation for further details).
\item {\tt FSRadiationMGPreRelax} [1] -- this gives the number of
  pre-relaxation sweeps that should be performed by the multigrid
  smoother. 
\item {\tt FSRadiationMGPostRelax} [1] -- this gives the number of
  post-relaxation sweeps that should be performed by the multigrid
  smoother. 
\item {\tt FSRadiationNGammaDot} [0.0] -- input parameter used by 
  {\tt FSProb\_RadiationSource.src90} to supply the function
  $\eta_f(\xvec)^n$. 
\item {\tt FSRadiationEtaRadius} [0.0] -- input parameter used by 
  {\tt FSProb\_RadiationSource.src90} to supply the function
  $\eta_f(\xvec)^n$. 
\item {\tt FSRadiationEtaCenter} [0.0 0.0 0.0] -- input parameters
  used by {\tt FSProb\_RadiationSource.src90} to supply the function
  $\eta_f(\xvec)^n$. 
\end{itemize}


\section{Concluding remarks}
\label{sec:conclusions}

We wish to remark that the module is not large (one header file, 8 C++
files, 3 F90 files), and all files begin with the {\tt FSProb}
prefix.  Most of the user interface is handled through the routines
{\tt FSProb\_Initialize.C} and {\tt FSProb\_RadiationSource.src90}, and
most of the 'action' may be found in the files {\tt FSProb\_Evolve.C}
and {\tt FSProb\_SetupSystem.src90}.  

Feedback/suggestions are appreciated.


\bibliography{sources}
\bibliographystyle{siam}
\end{document}

