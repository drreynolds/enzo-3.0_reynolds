\documentclass[letterpaper,10pt]{article}
\usepackage{geometry}   % See geometry.pdf to learn the layout
                        % options.  There are lots.
\usepackage[latin1]{inputenc}
\usepackage{graphicx}
\usepackage{epstopdf}
\usepackage{amsmath,amsfonts,amssymb}

\DeclareGraphicsRule{.tif}{png}{.png}{`convert #1 `dirname #1`/`basename #1 .tif`.png}

\author{Daniel R. Reynolds}
\title{{\tt gFLDProblem}: \\
A FLD-based Radiation and Chemistry Solver for ENZO}

\renewcommand{\(}{\left(}
\renewcommand{\)}{\right)}
\newcommand{\vb}{{\bf v}_b}
\newcommand{\xvec}{{\bf x}}
\newcommand{\Omegabar}{\bar{\Omega}}
\newcommand{\rhob}{\rho_b}
\newcommand{\dt}{\Delta t}
\newcommand{\Eot}{E^{OT}}
\newcommand{\Ef}{E_f}
\newcommand{\sighat}{\hat{\sigma}}
\newcommand{\Fnu}{{\bf F}_{\nu}}
\newcommand{\Pnu}{\overline{\bf P}_{\nu}}
\newcommand{\R}{I\!\!R}
\newcommand{\Rthree}{\R^3}
\newcommand{\eh}{e_h}
\newcommand{\ec}{e_c}
\newcommand{\Edd}{\mathcal F}
\newcommand{\Eddnu}{\Edd_{\nu}}
\newcommand{\mn}{{\tt n}}
\newcommand{\mB}{\mathcal B}
\newcommand{\mC}{{\mathcal C}}
\newcommand{\mL}{{\mathcal L}}
\newcommand{\mD}{{\mathcal D}}
\newcommand{\mDnu}{\mD_{\nu}}
\newcommand{\mCnu}{\mC_{\nu}}
\newcommand{\mLnu}{{\mathcal L}_{\nu}}
\newcommand{\mCe}{\mC_e}
\newcommand{\mLe}{\mL_e}
\newcommand{\mCn}{\mC_{\mn}}
\newcommand{\mLn}{\mL_{\mn}}


\textheight 9truein
\textwidth 6.5truein
\addtolength{\oddsidemargin}{-0.25in}
\addtolength{\evensidemargin}{-0.25in}
\addtolength{\topmargin}{-0.5in}
\setlength{\parindent}{0em}
\setlength{\parskip}{2ex}


\begin{document}
\maketitle

\section{Introduction}
\label{sec:intro}

This document is meant to describe a new, highly scalable,
field-based radiation and chemistry solver solver for Enzo, 
{\tt gFLProblem}. The target applications of this solver include the
transport of radiation using a flux-limited-diffusion-based solver
over a uniform grid, distributed among a possibly large number of
processors.  In this solver, the radiation field is assumed to be
either a monochromatic radiation energy at the ionization threshold of
Hydrogen I ($h\nu = 13.6$ eV), or an integrated radiation energy
density with an assumed radiation spectrum.  This radiation field may
be coupled to the surrounding matter using either an assumption of
local thermodynamic equilibrium (no chemical ionization), or a
Hydrogen-only universe using a case-B Hydrogen II recombination
coefficient.  

The defining characteristic of this solver in comparison 
with the {\tt gFLDSplit} solver is that in a given time step, this
solver evolves the radiation, a gas energy correction, and possibly a
Hydrogen-I number density together, using a full nonlinearly-implicit
solver.  The goal in using a full nonlinear solver on this problem is
to self-consistently capture the stiff interactions between these
processes in a time-accurate and stable fashion. 

This guide will only highlight the solvers and equations available in
this module.  For further details on the equations, numerical methods,
and verification tests relevant to this module, we refer to the paper 
\cite{ReynoldsHayesPaschosNorman2009}.




\section{Flux-limited diffusion radiation model}
\label{sec:rad_model}

We begin with the equation for flux-limited diffusion radiative
transfer in a cosmological medium \cite{ReynoldsHayesPaschosNorman2009},
\begin{equation}
\label{eq:radiation_PDE}
  \partial_{t} E + \frac1a \nabla\cdot\(E\vb\) =
    \nabla\cdot\(D\,\nabla E\) - \frac{\dot{a}}{a} E - c\kappa E + 4\pi\eta,
\end{equation}
where here the comoving radiation energy density $E$ and emissivity
$\eta$ are functions of space and time.  In this equation, the
frequency-dependence of the radiation energy has been integrated away,
under the premise of an assumed radiation energy spectrum,
\begin{align}
\label{eq:spectrum}
  & E_{\nu}(\nu,\xvec,t) = E(\xvec,t) \chi(\nu), \\
  \Rightarrow & \\
  & E(\xvec,t) = \(\int_{\nu_0}^{\infty} E_{\nu}(\nu,\xvec,t)\,\mathrm{d}\nu \)
    \Big/ \( \int_{\nu_0}^{\infty} \chi(\nu)\,\mathrm{d}\nu \).
\end{align}
As such, the emissivity function $\eta(\xvec,t)$
relates to the true emissivity $\eta_{\nu}(\nu,\xvec,t)$ as
\begin{equation}
\label{eq:emissivity}
  \eta(\xvec,t) = \( \int_{\nu_0}^{\infty}\eta_{\nu}(\nu,\xvec,t)\,\mathrm{d}\nu \) 
  \Big/ \( \int_{\nu_0}^{\infty}\chi(\nu)\,\mathrm{d}\nu \).
\end{equation}

The function $D$ in the above equation \eqref{eq:radiation_PDE} is
the {\em flux-limiter} that depends on $E$, $\nabla E$ and the 
opacity $\kappa$ (see
\cite{HayesNorman2003,ReynoldsHayesPaschosNorman2009}),  
\begin{equation}
\label{eq:limiter}
\begin{split}
   D(E) &= \text{diag}\( D_1(E),\, D_2(E),\, D_3(E) \) \\
   \text{where}\qquad & \\
   D_i(E) &= \frac{c(2\kappa+R_i)}{6\kappa^2+3\kappa R_i+R_i^2}, \\
   R_i &= \frac{|\partial_i E|}{E},\, i=1,2,3.
\end{split}
\end{equation}


\section{LTE couplings}
\label{sec:lte_model}


\section{Chemistry-dependent couplings}
\label{sec:chem_model}


\section{Numerical solution approach}
\label{sec:solution_approach}





\section{{\tt gFLDProblem} usage}
\label{sec:module_usage}

In order to use the implicit FLD radiation solver module, and to
allow optimal control over the nonlinear and linear solution methods
used, there are a number of parameters than may be supplied to Enzo.
We group these into two categories, those associated with the general
startup of the module via the Enzo infrastructure, and those that may
be supplied to the module itself.  However, prior to embarking on a
description of these parameters, there are a few requirements for any
problem that wishes to use the {\tt gFLDProblem} module.

Foremost, Enzo must be built using the configuration option 
{\tt RAD\_HYDRO}; this is enabled through the call 
{\tt gmake rad-hydro-yes} in the Enzo source directory.  Moreover, the
machine Makefile must specify how to include and link with an
available HYPRE library (version $\ge$ 2.4.0b).  If a user must
compile HYPRE themselves to obtain this version, they should make note
of the HYPRE configuration {\tt --with-no-global-partition}, which
must be used for solver scalability when using over $\sim1000$
processors, but which results in slower executables on smaller
problems.


\subsection{Startup parameters}

In a user's main problem parameter file, the following three parameters
must be set:
\begin{itemize}
\item {\tt RadiationHydrodynamics} -- this must be set to either 1 or
  2.  A value of 1 enables the use of implicit radiation solvers
  alongside standard Enzo chemistry and hydrodynamics physics.  A
  value of 2 turns on the implicit radiation solvers, and {\em turns off}  
  all other Enzo physics solvers.
\item {\tt ImplicitProblem} -- this must be set to 2 to use the 
  {\tt FSProb} module (among the possible implicit radiation solver
  modules).
\item {\tt RadHydroParamfile} -- this should contain the filename
  (relative to this parameter file) that contains all module-specific
  solver parameters.  While the {\tt FSProb} module parameters may be
  supplied in the main parameter file, it is not recommended since
  Enzo's main {\tt ReadParameterFile.C} routine will complain about
  all of the 'unknown' parameters.
\end{itemize}

In addition, in a user's problem initialization routines, they must
allocate a standard Enzo baryon field having the {\tt FieldType} set
to {\tt RadiationFreq0}.  It is this baryon field that will be evolved
by the {\tt FSProb} module, and that a user may access to obtain
information on the free-streaming radiation field.

Furthermore, the {\tt FSProb} module currently computes its own
emissivity field $\eta_f(\xvec)^n$, in the routine 
{\tt FSProb\_RadiationSource.src90}.  A user may edit this file to add
in an emissivity field corresponding to their own Enzo 
{\tt ProblemType}.  Alternatively, we are currently working to
implement a separate interface whereby some other Enzo function fills
in an emissivity field, which is then used internally by the 
{\tt FSProb} module.

For convenience, we have provided the {\tt ProblemType} 250, with the
initialization filed {\tt FSMultiSourceInitialize.C} and 
{\tt Grid\_FSMultiSourceInitializeGrid.C}, as an example of how to set
up this module.



\subsection{Module parameters}

Once a user has enabled the free-streaming radiation module, they have
complete control over a variety of internal module parameters.  The
parameters are given here, with their default values specified in brackets.
\begin{itemize}
\item {\tt FSRadiationScaling} [1.0] -- this parameter may be used to
  rescale the internal solver units for the free-streaming radiation
  field, to ensure that the linear system is well-scaled to have
  solution with values on the order of 1.  We note that without
  scaling, the solver will use the internal Enzo units for a radiation
  field, i.e.~the radiation field will have values in 
  CGS/({\tt DensityUnits*VelocityUnits*VelocityUnits}). 
\item {\tt FSRadiationTheta} [1.0] -- this is the implicit
  discretization method parameter.  1 gives implicit Euler, and 2
  gives the trapezoid rule (Crank-Nicolson).
\item {\tt FSRadiationOpacity} [$10^{-4}$] -- this is the opacity
  constant used within the computations of the flux-limiter
  $D_f(\Ef)$.  The expected values should be those of a (small)
  physical opacity for the problem in CGS.
\item {\tt FSRadiationLimiterType} [4] -- this gives the formula for
  the flux limiter to be used in the code.  The default value provides
  the formula \eqref{eq:limiter}; other allowable values and their
  formulas may be found in the file {\tt FSProb\_SetupSystem.src90}.
\item {\tt FSRadiationBoundaryX0Faces} [0 0] -- these integer
  constants give the type of boundary conditions to supply on the
  free-streaming radiation field at the lower and upper domain
  boundaries in the x-direction, respectively.  Allowable 
  constants are 0 (periodic), 1 (Dirichlet), and 2 (Neumann).  If
  either 1 or 2 are provided, the file {\tt FSProb\_Initialize} will
  need to be modified to set the Dirichlet or Neumann values on the
  appropriate faces, based on the {\tt ProblemType}.
\item {\tt FSRadiationBoundaryX1Faces} [0 0] -- lower and upper
  boundary condition types for the y-direction.
\item {\tt FSRadiationBoundaryX2Faces} [0 0] -- lower and upper
  boundary condition types for the z-direction.
\item {\tt FSRadiationMaxDt} [0.0] -- this gives a maximum allowed
  time step size for the free-streaming radiation problem.  The
  default value of 0 turns off control over the maximum step size; any
  positive value will turn it on.
\item {\tt FSRadiationInitialGuess} [0] -- this gives control over the
  algorithm to use in computing the initial guess at the time-evolved
  solution.  Since the solver is iterative, a good initial guess will
  result in faster convergence.  Allowable values may be found in the
  file {\tt FSProb\_InitialGuess.src90}.
\item {\tt FSRadiationTolerance} [$10^{-10}$] -- this gives the
  absolute tolerance to be used in solving the linear system
  \eqref{eq:fs_linear}.  Note, this absolute tolerance is used on the
  scaled linear system (see {\tt FSRadiationScaling}).
\item {\tt FSRadiationMaxMGIters} [50] -- this gives the maximum
  number of multigrid v-cycles to be used.
\item {\tt FSRadiationMGRelaxType} [1] -- this gives the type of
  smoother to use within multigrid v-cycles.  Values of 0 or 1 give
  weighted Jacobi, and values of 2 or 3 give red-black Gauss-Seidel
  (see HYPRE documentation for further details).
\item {\tt FSRadiationMGPreRelax} [1] -- this gives the number of
  pre-relaxation sweeps that should be performed by the multigrid
  smoother. 
\item {\tt FSRadiationMGPostRelax} [1] -- this gives the number of
  post-relaxation sweeps that should be performed by the multigrid
  smoother. 
\item {\tt FSRadiationNGammaDot} [0.0] -- input parameter used by 
  {\tt FSProb\_RadiationSource.src90} to supply the function
  $\eta_f(\xvec)^n$. 
\item {\tt FSRadiationEtaRadius} [0.0] -- input parameter used by 
  {\tt FSProb\_RadiationSource.src90} to supply the function
  $\eta_f(\xvec)^n$. 
\item {\tt FSRadiationEtaCenter} [0.0 0.0 0.0] -- input parameters
  used by {\tt FSProb\_RadiationSource.src90} to supply the function
  $\eta_f(\xvec)^n$. 
\end{itemize}


\section{Concluding remarks}
\label{sec:conclusions}

We wish to remark that the module is not large (one header file, 8 C++
files, 3 F90 files), and all files begin with the {\tt FSProb}
prefix.  Most of the user interface is handled through the routines
{\tt FSProb\_Initialize.C} and {\tt FSProb\_RadiationSource.src90}, and
most of the 'action' may be found in the files {\tt FSProb\_Evolve.C}
and {\tt FSProb\_SetupSystem.src90}.  

Feedback/suggestions are appreciated.


\bibliography{sources}
\bibliographystyle{siam}
\end{document}
