\documentclass[letterpaper,10pt]{article}
\usepackage{geometry}   % See geometry.pdf to learn the layout
                        % options.  There are lots.
\usepackage[latin1]{inputenc}
\usepackage{graphicx}
\usepackage{epstopdf}
\usepackage{amsmath,amsfonts,amssymb}

\DeclareGraphicsRule{.tif}{png}{.png}{`convert #1 `dirname #1`/`basename #1 .tif`.png}

\author{Daniel R. Reynolds}
\title{{\tt gFLDProblem}: \\
A FLD-based Radiation and Chemistry Solver for Enzo}

\renewcommand{\(}{\left(}
\renewcommand{\)}{\right)}
\newcommand{\vb}{{\bf v}_b}
\newcommand{\xvec}{{\bf x}}
\newcommand{\Omegabar}{\bar{\Omega}}
\newcommand{\rhob}{\rho_b}
\newcommand{\dt}{\Delta t}
\newcommand{\Eot}{E^{OT}}
\newcommand{\Ef}{E_f}
\newcommand{\sighat}{\hat{\sigma}}
\newcommand{\Fnu}{{\bf F}_{\nu}}
\newcommand{\Pnu}{\overline{\bf P}_{\nu}}
\newcommand{\R}{I\!\!R}
\newcommand{\Rthree}{\R^3}
\newcommand{\eh}{e_h}
\newcommand{\ec}{e_c}
\newcommand{\Edd}{\mathcal F}
\newcommand{\Eddnu}{\Edd_{\nu}}
\newcommand{\mn}{{\tt n}}
\newcommand{\mB}{\mathcal B}
\newcommand{\mC}{{\mathcal C}}
\newcommand{\mL}{{\mathcal L}}
\newcommand{\mD}{{\mathcal D}}
\newcommand{\mDnu}{\mD_{\nu}}
\newcommand{\mCnu}{\mC_{\nu}}
\newcommand{\mLnu}{{\mathcal L}_{\nu}}
\newcommand{\mCe}{\mC_e}
\newcommand{\mLe}{\mL_e}
\newcommand{\mCn}{\mC_{\mn}}
\newcommand{\mLn}{\mL_{\mn}}


\textheight 9truein
\textwidth 6.5truein
\addtolength{\oddsidemargin}{-0.25in}
\addtolength{\evensidemargin}{-0.25in}
\addtolength{\topmargin}{-0.5in}
\setlength{\parindent}{0em}
\setlength{\parskip}{2ex}


\begin{document}
\maketitle

\section{Introduction}
\label{sec:intro}

This document describes a new, highly scalable,
field-based radiation and chemistry solver for Enzo, 
{\tt gFLProblem}. The target applications of this solver include the
transport of radiation using a flux-limited-diffusion-based solver,
distributed among a possibly large number of 
processors.  In this solver, the radiation field is assumed to be
either a monochromatic radiation energy at the ionization threshold of
Hydrogen I ($h\nu = 13.6$ eV), or an integrated radiation energy
density with an assumed radiation spectrum.  This radiation field may
be coupled to the surrounding matter using either an assumption of
local thermodynamic equilibrium (no chemical ionization), or a model
for Hydrogen ionization. 

The defining characteristic of this solver in comparison 
with {\tt gFLDSplit} is that in a given time step, this
solver evolves the radiation, a gas energy correction, and possibly a
Hydrogen-I number density together, using a full nonlinearly-implicit
solver.  The goal in using a nonlinear solver on this problem is
to self-consistently capture the stiff interactions between these
processes in a time-accurate and numerically stable fashion.  It is
therefore usually slower than its sister solver, {\tt gFLDSplit},
which takes more shortcuts in numerical accuracy to obtain a faster
and more robust solver for a similar physical regime.  In addition to
the full nonlinear solver, this module allows increased choice over 
modeling parameters, as it is used as a test-bed for developing new
methods. 

This guide will only highlight the solvers and equations available in
this module.  For further details on the equations, numerical methods,
and verification tests relevant to this module, we refer to the paper 
\cite{ReynoldsHayesPaschosNorman2009}.  Moreover, we start right in
with the relevant parameters that must be defined to use the module.
The equations that describe what these options mean follow in
subsequent sections, with references placed accordingly.



\subsection{Current limitations}
\label{sec:limitations}

This solver is still a work in progress, though will serve many user's
needs in its current form.  Specific limitations that are present in
the current solver, and that are under active development include:
\begin{itemize}
\item Adaptive mesh refinement -- This solver may currently only be
  run in unigrid simulations.  This is because in the implicit
  radiation equation we must solve the problem on the entire mesh
  hierarchy, and the linear solver interface changes (and becomes more
  complex) when moving to a hierarchical mesh structure.  We have
  finished an initial interface for linear solvers on hierarchical
  meshes, in the context of self-gravity solves.  We plan to extend
  this interface to the implicit {\tt gFLDProblem} module in the near
  future.
\item Increased chemistry -- Due to the use of a single radiation
  field with an assumed spectrum, our initial solver development
  focused only on Hydrogen chemistry.  This was based on the fact that
  multiple chemical species would be best simulated with a radiation
  approximation that allows spectral variation in space, to allow for
  spectral hardening and I-front preheating.  However, we plan on
  extending this single-field solver to work with Helium and molecular
  chemistry, for problems where spectral variation may be less
  important.
\end{itemize}




\section{{\tt gFLDProblem} usage}
\label{sec:module_usage}

In order to use the implicit FLD radiation solver module, and to
allow optimal control over the nonlinear and linear solution methods
used, there are a number of parameters than may be supplied to Enzo.
We group these into two categories, those associated with the general
startup of the module via the Enzo infrastructure, and those that may
be supplied to the module itself.  However, prior to embarking on a
description of these parameters, there are a few requirements for any
problem that wishes to use the {\tt gFLDProblem} module.

Foremost, Enzo must be built using the two configuration options
{\tt PHOTON} (enabled through the call {\tt gmake photon-yes} in the
Enzo source directory), and {\tt HYPRE} (enabled using the call 
{\tt gmake hypre-yes}).  Moreover, the machine Makefile must specify
how to include and link with an available HYPRE library (version $\ge$
2.4.0b).  If a user must compile HYPRE themselves to obtain this
version, they should make note of the HYPRE configuration option
{\tt --with-no-global-partition}, which must be used for solver
scalability when using over $\sim\!1000$ processors, but which results
in slower executables on smaller-scale problems.


\subsection{Startup parameters}

In a user's main problem parameter file, the following parameters
must be set (their default values are in brackets):
\begin{itemize}
\item {\tt RadiativeTransferFLD} [0] -- this must be set to 2.  Other
  values will either disable the FLD solver, or will use one in a
  non-desired context.
\item {\tt ImplicitProblem} [0] -- this must be set to 1 to use this
  {\tt gFLDProblem} module (among the possible implicit radiation solver
  modules).
\item {\tt ProblemType} [0] -- as usual, this is problem-dependent.
  However, for implicit FLD-based solvers, the value of ProblemType
  should be within the 400's.
\item {\tt RadHydroParamfile} [NULL] -- this should contain the filename
  (with path relative to this parameter file) that contains all
  module-specific solver parameters (discussed below).  While the 
  {\tt gFLDProblem} module parameters may be supplied in the main
  parameter file, that filename must still be specified here (though
  it is not recommended, since the {\tt ReadParameterFile.C} routine
  will complain about all of the `unknown' parameters that are read
  elsewhere).
\item {\tt RadiationFieldType} [0] -- this can be any value {\em except}
  10 or 11, since those use pre-existing background radiation
  approximations.
\item {\tt RadiativeTransferFLDCallOnLevel} [0] -- this should currently
  be set to 0.  Future releases plan to enable the implicit solver on
  a statically nested subgrid, but this does not work at present.
\item {\tt RadiativeTransfer} [0] -- this must be set to 0.  A nonzero
  value will instead use an explicit ray-tracing solver for the
  radiation transport. 
\item {\tt RadiativeTransferOpticallyThinH2} [1] -- this must be set
  to 0.  A nonzero value will attempt to use a $1/r^2$ Lyman-Werner
  radiation field, that is {\em ignored} by the {\tt gFLDProblem} module.
\item {\tt RadiativeCooling} [0] -- this must be set to 0.  A nonzero
  value will attempt to use Enzo's built-in explicit subcycled gas
  cooling modules instead of the coupled radiative-ionization provided
  by {\tt gFLDProblem}, which has not yet been enabled to work together. 
\item {\tt GadgetEquilibriumCooling} [0] -- this must be set to 0.  A
  nonzero value will attempt to use Enzo's built-in equilibrium cooling
  modules instead of the coupled radiative-ionization provided by 
  {\tt gFLDProblem}, which have also not been set up to work together. 
\end{itemize}

In addition, if a user wishes to set up a new {\tt ProblemType} that
uses the {\tt gFLDProblem} module, they must allocate a standard Enzo
baryon field having the {\tt FieldType} set to {\tt RadiationFreq0}.
It is this baryon field that will be evolved by the {\tt gFLDProblem}
module, and that a user may access to obtain information on the
grey radiation field. 

Furthermore, the {\tt gFLDProblem} module currently computes its own
emissivity field $\eta(\xvec)$, in the routine 
{\tt gFLDProblem\_RadiationSource.src90}.  A user may edit this file
to add in an emissivity field corresponding to their own 
{\tt ProblemType}.  Alternatively, a user may separately fill in the
BaryonField {\tt Emissivity0} in some other Enzo routine.  Then, when
Enzo is compiled using the pre-processor directive {\tt EMISSIVITY}
and run with the global parameter {\tt StarMakerEmissivity $\ne$ 0},
the {\tt gFLDProblem} module will use the {\tt Emissivity0} field
instead of computing its own. 



\subsection{Module parameters}

Once a user has enabled the {\tt gFLDProblem} module, they have
complete control over a variety of internal module parameters.  The
parameters are given here, with their default values specified in
brackets, and references to the appropriate equations elsewhere in
this document. 
\begin{itemize}
\item {\tt RadHydroESpectrum} [1], this parameter chooses the type of
  assumed radiation energy spectrum from equation \eqref{eq:spectrum}.
  Allowed values include:
  \begin{itemize}
  \item[-1.] monochromatic spectrum at frequency $h\nu_{HI} = 13.6$ eV.
  \item[0.] power law spectrum,
    \[
      \chi(\nu) = \left(\frac{\nu}{\nu_{HI}}\right)^{-1.5}
    \]
  \item[1.] $T=10^5$ K blackbody spectrum, 
    \[
       \chi(\nu) = \frac{8 \pi h
         \left(\frac{\nu}{c}\right)^3}{\exp\left(\frac{h\nu}{k_b 10^5}\right)-1}.
    \]
  \end{itemize}
\item {\tt RadHydroChemistry} [1], this parameter controls whether to
  use Hydrogen chemistry or not.  Allowable values are:
  \begin{itemize}
  \item[0.] no chemistry
  \item[1.] Hydrogen chemistry
  \end{itemize}
\item {\tt RadHydroHFraction} [1], this parameter controls the
  fraction of baryonic matter comprised of Hydrogen, allowable
  values are $0 \le {\tt RadHydroHFraction} \le 1$.
\item {\tt RadHydroModel} [1], this parameter determines which model
  for radiation-matter coupling we wish to use, allowable values
  include:
  \begin{itemize}
  \item[1.] Use the chemistry-dependent model from section
    \ref{sec:chem_model}, with a case B HII recombination coefficient.
  \item[2.] Use the chemistry-dependent model from section
    \ref{sec:chem_model}, with a case A HII recombination coefficient.
  \item[4.] The same as model 1, but assume an isothermal gas energy
    (for regression test problems).
  \item[10.] Use the local thermodynamic equilibrium model from section
    \ref{sec:lte_model}.
  \end{itemize}
\item {\tt RadHydroMaxDt} [$10^{20}$], this parameter sets the value of
  $\dt_{\text{max}}$ from section \ref{sec:dt_selection}; it must be
  greater than 0.  This value must be given in {\em scaled} time units,   
  i.e.~$\dt_{\text{physical}} \le \dt_{\text{max}}*\text{TimeUnits}$, 
  where TimeUnits is Enzo's internal time scaling factor for the simulation.
\item {\tt RadHydroMinDt} [0], this parameter sets the value of
  $\dt_{\text{min}}$ from section \ref{sec:dt_selection}; it must be
  non-negative.  This value must also be given in scaled time
  units.
\item {\tt RadHydroInitDt} [$10^{20}$], this parameter sets the initial
  time step size for the {\tt gFLDProblem} module.  We note that since
  the module will take the smaller of $\dt_{\text{FLD}}$ and
  $\dt_{\text{CFL}}$, the default value is never actually used.  This
  value must also be given in scaled time units.
\item {\tt RadHydroDtNorm} [2.0], this parameter sets the value
  of $p$ from equation \eqref{eq:time_error}.  
  \begin{itemize}
  \item A value of $0$ implies to use the $\max$ norm, 
  \item A value $>0$ implies to use the corresponding $p$-norm,
  \item Values $<0$ are not allowed.
  \end{itemize}
\item {\tt RadHydroDtRadFac}, {\tt RadHydroDtGasFac} and {\tt
    RadHydroDtChemFac}  [$10^{20}$], these parameters give the values
  of $\tau_{\text{tol}}$ for the variables $E$, $e_c$ and $\mn_{HI}$
  from equation \eqref{eq:time_estimate}, respectively.  They must be
  positive; the default specifies no restrictions on $\dt_{\text{FLD}}$.
\item {\tt RadiationScaling}, {\tt EnergyCorrectionScaling} and 
  {\tt ChemistryScaling} [1.0], these parameters give the scaling 
  factors $s_E$, $s_e$ and $s_{\mn}$ from
  \eqref{eq:variable_rescaling}, respectively; supplied values must be
  positive. 
\item {\tt RadiationBoundaryX0Faces}, {\tt RadiationBoundaryX1Faces}
  and {\tt RadiationBoundaryX2Faces} [0 0], these specify the
  boundary-condition types from section \ref{sec:boundary_conditions}
  to use on the lower and upper boundaries in each direction.
  Allowable values are:
  \begin{itemize}
  \item[0.] periodic (must match on both faces in a given direction)
  \item[1.] Dirichlet
  \item[2.] Neumann
  \end{itemize}
\item {\tt RadHydroLimiterType} [4], this parameter determines the
  type of flux limiter $D$ to use in \eqref{eq:radiation_PDE}:
  \begin{itemize}
  \item[0.] specifies the original Levermore-Pomraning limiter,
    \eqref{eq:LP_limiter} \cite{Levermore1984,LevermorePomraning1981}.
  \item[1.] specifies the rational approximation to the
    Levermore-Pomraning limiter, \eqref{eq:ratLP_limiter}.
  \item[2.] specifies to use the Larsen $n=2$ flux limiter,
    \eqref{eq:Larsen_n2_limiter} \cite{Morel2000}. 
  \item[3.] specifies to use no limiter, \eqref{eq:no_limiter}.
  \item[4.] specifies to use the ZEUS limiter, \eqref{eq:Zeus_limiter}.
  \end{itemize}
\item {\tt RadHydroTheta} [1.0], this parameter specifies the
  value of $\theta$ in equations
  \eqref{eq:radiation_PDE_theta}-\eqref{eq:hydrogen_theta},
  $0\le\theta\le 1$.
\item {\tt RadHydroAnalyticChem} [1], this parameter specifies to use
  the implicit quasi-steady-state time approximation, i.e.~equations
  \eqref{eq:radiation_PDE_iqss}-\eqref{eq:hydrogen_iqss}, instead of
  the $\theta$-method in time.  Allowable values are $\{0,1\}$, and a
  value of 1 will be used for all models for which this has been
  implemented (all models allow the $\theta$-method).
\item {\tt RadHydroInitialGuess} [0], this parameter specifies the
  method to use in calculating an initial guess to the time-evolved
  nonlinear problem.  The allowed values are the same as those
  described in section \ref{sec:initial_guess}, with 0 and 5 being the
  safest to use.
\item {\tt RadHydroNewtTolerance} [$10^{-6}$], this parameter
  specifies the nonlinear tolerance $\varepsilon$ from section
  \ref{sec:nkmg}.  Allowable values must be between $10^{-15}$ and 1.
\item {\tt RadHydroNewtIters} [20], this positive parameter specifies
  the maximum allowed number of Inexact Newton iterations, as
  described in section \ref{sec:nkmg}.  If the nonlinear problem has
  not been solved after this many iterations, the solver will return
  with a failure. 
\item {\tt RadHydroINConst} [$10^{-8}$], this parameter specifies
  the value of $C$ used in determining $\delta_k$ within the Inexact
  Newton iteration from section \ref{sec:nkmg}.  Allowable values must
  be between 0 and 1.
\item {\tt RadHydroMaxMGIters} [50], this positive parameter
  specifies the maximum number of multigrid iterations to perform in
  the MG-CG solver from section \ref{sec:nkmg}.
\item {\tt RadHydroMGRelaxType} [1], this parameter specifies the
  relaxation method used by the multigrid solver:
  \begin{itemize}
  \item[0.] Jacobi
  \item[1.] Weighted Jacobi
  \item[2.] Red/Black Gauss-Seidel (symmetric)
  \item[3.] Red/Black Gauss-Seidel (nonsymmetric)
  \end{itemize}
  For more information, see the HYPRE user manual.
\item {\tt RadHydroMGPreRelax} [1], this positive parameter
  specifies the number of pre-relaxation sweeps the multigrid solver
  should use in the MG-CG solver from section \ref{sec:nkmg}.
\item {\tt RadHydroMGPostRelax} [1], this positive parameter
  specifies the number of post-relaxation sweeps the multigrid solver
  should use in the MG-CG solver from section \ref{sec:nkmg}.
\item {\tt EnergyOpacityC0}-{\tt EnergyOpacityC4} [1, 1, 0, 1, 0],
  these specify the opacity-defining constants $C_0$-$C_4$ for the
  energy-mean opacity $\kappa_E$ as in equation \eqref{eq:opacityE}.
  The parameters $C_0$, $C_2$ and $C_4$ must all be $\ge 0$, while the
  parameters $C_1$ and $C_3$ must be $> 0$.
\item {\tt PlanckOpacityC0}-{\tt PlanckOpacityC4} [1, 1, 0, 1, 0], the
  same as above, but for the Planck-mean opacity $\kappa_P$.
\end{itemize}





\section{Flux-limited diffusion radiation model}
\label{sec:rad_model}

We begin with the equation for flux-limited diffusion radiative
transfer in a cosmological medium \cite{ReynoldsHayesPaschosNorman2009},
\begin{equation}
\label{eq:radiation_PDE}
  \partial_{t} E + \frac1a \nabla\cdot\(E\vb\) =
    \nabla\cdot\(D\,\nabla E\) - \frac{\dot{a}}{a} E - c\kappa E + 4\pi\eta,
\end{equation}
where here the comoving radiation energy density $E$, emissivity
$\eta$ and opacity $\kappa$ are functions of space and time.  In this
equation, the frequency-dependence of the radiation energy has been
integrated away, under the premise of an assumed radiation energy
spectrum, 
\begin{align}
  \notag
  & E_{\nu}(\nu,\xvec,t) = \tilde{E}(\xvec,t) \chi(\nu), \\
  \notag
  \Rightarrow & \\
  \label{eq:spectrum}
  & E(\xvec,t) = \int_{\nu_{HI}}^{\infty} E_{\nu}(\nu,\xvec,t)\,\mathrm{d}\nu 
    = \tilde{E}(\xvec,t) \int_{\nu_{HI}}^{\infty} \chi(\nu)\,\mathrm{d}\nu,
\end{align}
where $\tilde{E}$ is an intermediate quantity (for analysis) that is
never computed.  We note that if the assumed spectrum is the Dirac
delta function, $\chi(\nu) = \delta_{\nu_{HI}}(\nu)$, $E$ is a
monochromatic radiation energy density at the ionization threshold of
HI, and the $-\frac{\dot{a}}{a}E$ term (obtained through integration
by parts of the redshift term
$\frac{\dot{a}}{a}\partial_{\nu}E_{\nu}$) is omitted from
\eqref{eq:radiation_PDE}. Similarly, the emissivity function
$\eta(\xvec,t)$ relates to the true emissivity
$\eta_{\nu}(\nu,\xvec,t)$ by the formula
\begin{equation}
\label{eq:emissivity}
  \eta(\xvec,t) = \int_{\nu_{HI}}^{\infty}\eta_{\nu}(\nu,\xvec,t)\,\mathrm{d}\nu.
\end{equation}

The function $D$ in the above equation \eqref{eq:radiation_PDE} is
the {\em flux-limiter} that depends on $E$, $\nabla E$ and the 
opacity $\kappa$ (see
\cite{HayesNorman2003,ReynoldsHayesPaschosNorman2009}),  
\[
   D(E) = \text{diag}\( D_1(E),\, D_2(E),\, D_3(E) \),
\]
where the directional limiters $D_i(E)$ are chosen to be one of 
\begin{align}
  \label{eq:ratLP_limiter}
   D_i(E) &= \frac{c(2\kappa+R_i)}{\omega(6\kappa^2+3\kappa R_i + R_i^2)}, \\
  \label{eq:Larsen_n2_limiter}
   D_i(E) &= \frac{c}{\left((3\kappa)^2 + R_i^2\right)^{1/2}},  \\
  \label{eq:no_limiter}
   D_i(E) &= \frac{c}{3\kappa}, \\
  \label{eq:Zeus_limiter}
   D_i(E) &= \frac{c(2\kappa+R_i)}{6\kappa^2+3\kappa R_i+R_i^2}, \\
  \label{eq:LP_limiter}
   D_i(E) &= \frac{c \tanh\left(\frac{R_i}{\kappa}\right)-c \frac{\kappa}{R_i}}{\omega R_i}.
\end{align}
In all of the above, the ``effective albedo'' is given by
\begin{equation}
\label{eq:albedo}
  \omega = \frac{(4 \sigma_{SB} T^4)}{c E},
\end{equation}
and the limiter factor $R_i$ is given by either
\begin{align}
  \label{eq:R_zeus}
   R_i &= \max\left\{\frac{|\partial_i E|}{\omega E},
     10^{-20}\right\}, \qquad\text{[Levermore-Pomraning limiters,
     \eqref{eq:ratLP_limiter} \& \eqref{eq:LP_limiter}]}, \\
   R_i &= \max\left\{\frac{|\partial_i E|}{E}, 10^{-20}\right\},
     \qquad\text{[all others]}.
\end{align}



\section{LTE couplings}
\label{sec:lte_model}

For problems in which chemical ionization is unimportant, we may
assume that the gas is in local thermodynamic equilibrium.  In this
case we do not need to couple the radiation energy to a model for
chemical ionization, we therefore couple the radiation to the specific
gas energy equation, 
\begin{align}
  \label{eq:cons_energy}
  \partial_t e + \frac1a\vb\cdot\nabla e &=
    - \frac{2\dot{a}}{a}e
    - \frac{1}{a\rhob}\nabla\cdot\left(p\vb\right) 
    - \frac1a\vb\cdot\nabla\phi + G - \Lambda.
\end{align}
All but the final two terms in \eqref{eq:cons_energy} are already
handled by Enzo's existing hydrodynamics solver infrastructure.  In
this module, we therefore consider a specific energy correction equation 
\begin{align}
  \label{eq:cons_energy_correction}
  \partial_t e_c &= -\frac{2\dot{a}}{a}e_c + G - \Lambda,
\end{align}
that we use to correct Enzo's original gas energy to include radiation
couplings.  Here $G$ is the local heating rate,
\begin{align}
\label{eq:G_LTE}
  G &= \frac{c \kappa}{\rhob} E,
\end{align}
and $\Lambda$ corresponds to the local cooling rate,
\begin{align}
\label{eq:Lambda_LTE}
  \Lambda = \frac{4\pi}{\rhob} \eta.
\end{align}
The user-defined energy mean opacity $\kappa$ is
spatially-homogeneous, and is given by the formula
\begin{align}
\label{eq:opacityE}
  \kappa = C_0 \left(\frac{\rhob}{C_1}\right)^{C_2}
    \left(\frac{T}{C_3}\right)^{C_4} 
\end{align}
(the constants $C_0\to C_4$ are input by the user), and $\eta$ is
a black-body emissivity given by 
\begin{align}
\label{eq:etaBB}
  \eta = \frac{\kappa_P\,\sigma_{SB}\,T^4}{\pi},
\end{align}
where the user-defined Planck-mean opacity $\kappa_P$ is defined
similarly to the energy-mean opacity \eqref{eq:opacityE},
$\sigma_{SB}$ is the Stefan-Boltzmann constant [$5.6704\times 10^{-5}$
erg s$^{-1}$ cm$^{-2}$ K$^{-4}$], and $T$ is the gas temperature [K].



\section{Chemistry-dependent couplings}
\label{sec:chem_model}

In general, radiation calculations in Enzo are used in simulations
where chemical ionization states are important.  For these situations, 
we couple the radiation equation \eqref{eq:radiation_PDE} with
equations for both the specific gas energy correction and the
ionization dynamics of Hydrogen,
\begin{align}
  \notag
  \partial_t e_c &= -\frac{2\dot{a}}{a}e_c + G - \Lambda, \\
  \label{eq:hydrogen_ionization}
  \partial_t \mn_{HI} + \frac{1}{a}\nabla\cdot\(\mn_{HI}\vb\) &=
    \alpha^{rec} \mn_e \mn_{HII} - \mn_{HI} \Gamma_{HI}^{ph}. 
\end{align}
Here, $\mn_{HI}$ is the comoving Hydrogen I number density.  The
recombination rate $\alpha^{rec}$ may be chosen as either the case-B
recombination rate, 
\begin{equation}
\label{eq:alphaB}
\alpha^{rec} = 2.753\times 10^{-14} \left(\frac{3.15614\times 10^5}{T}\right)^{3/2} 
                   \left(1+\left(\frac{3.15614\times 10^5}{2.74\, T}\right)^{0.407}\right)^{-2.242} 
\end{equation}
or the case-A rate found in Enzo's chemistry rate lookup tables.

In this model, the gas heating and cooling rates are
chemistry-dependent, 
\begin{align}
  \label{eq:G_nLTE}
  G &= \frac{c\,E\,\mn_{HI}}{\rhob} 
    \left[\int_{\nu_{HI}}^{\infty} \sigma_{HI}\, \chi_E
    \left(1-\frac{\nu_{HI}}{\nu}\right)\, d\nu\right] \bigg/
    \left[\int_{\nu_{HI}}^{\infty} \chi_E d\nu\right], \\
\label{eq:Lambda_nLTE}
  \Lambda &= \frac{\mn_e}{\rhob}\bigg[\text{ce}_{HI}\, \mn_{HI} 
  + \text{ci}_{HI}\, \mn_{HI} + \text{re}_{HII}\, \mn_{HII} + \text{brem}\,
  \mn_{HII} \\
  \notag &\qquad+ \frac{m_h}{\rho_{units}\, a^3} \left(\text{comp}_1\, (T-\text{comp}_2) 
    + \text{comp}_{X}\, (T-\text{comp}_{T})\right) \bigg].
\end{align}
The temperature-dependent cooling rates
$\text{ce}_{HI}$, $\text{ci}_{HI}$, $\text{re}_{HII}$, $\text{brem}$,
$\text{comp}_1$, $\text{comp}_2$, $\text{comp}_{X}$ and
$\text{comp}_{T}$ are all taken from Enzo's built-in rate tables.

Moreover, the frequency-integrated opacity is now chemistry-dependent,
\begin{equation}
\label{eq:opacityHI}
  \kappa \ = \ 
  \left[\int_{\nu_{HI}}^{\infty} \kappa_{\nu}\,E_{\nu}\,d\nu\right] \bigg/
  \left[\int_{\nu_{HI}}^{\infty} E_{\nu}\,d\nu\right] \ = \ 
  \left[\mn_{HI} \int_{\nu_{HI}}^{\infty}
    \chi_E\,\sigma_{HI}\,d\nu\right] \bigg/
  \left[\int_{\nu_{HI}}^{\infty} \chi_E\,d\nu\right],
\end{equation}
where these integrals with the assumed radiation spectrum $\chi(\nu)$
handle the change from the original frequency-dependent radiation
equation to the integrated grey radiation equation. 




\section{Numerical solution approach}
\label{sec:solution_approach}

We solve these models in an implicit-time fashion, coupling 
\eqref{eq:radiation_PDE}, \eqref{eq:cons_energy_correction} and
possibly \eqref{eq:hydrogen_ionization} together to form a large
system of nonlinear PDEs that must be solved in each time step to find
the updated solution.  This solve is coupled to Enzo's existing
operator-split solver framework in the following manner:
\begin{itemize}
\item[(i)] Evolve the radiation, gas energy correction and Hydrogen I
  number density implicitly in time [{\tt gFLDProblem}].
\item[(ii)] Project the dark matter particles onto the finite-volume
  mesh to generate a dark-matter density field $\rho_{dm}$ [Enzo];
\item[(iii)] Solve for the gravitational potential $\phi$ using a
  Poisson equation [Enzo];
\item[(iv)] Advect the dark matter particles with the Particle-Mesh
  method [Enzo];
\item[(v)] Evolve the hydrodynamics equations using an up to
  second-order explicit method, and have the velocity $\vb$ advect
  both the Hydrogen I number density $\mn_{HI}$ and the grey radiation
  field $E$ [Enzo]; 
\end{itemize}

The implicit solution approach for step (i) is described in detail in 
\cite{ReynoldsHayesPaschosNorman2009}; here we describe only enough
to point out the available user parameters, and more fully describe
some additional options available in the solver.

\subsection{Time discretizations}
\label{sec:iqss}

We must first discretize the equations \eqref{eq:radiation_PDE},
\eqref{eq:cons_energy_correction} and \eqref{eq:hydrogen_ionization}
in space and time before we solve them computationally.  We use a
method of lines approach for the space-time discretization, wherein we
first discretize in space, and then evolve the resulting system of
ODEs in time.  As with the rest of Enzo, we use a finite-volume
spatial discretization, placing all of our unknowns at the center of
each finite-volume cell, and performing all spatial derivatives
through a divergence of face-centered fluxes.  

However, this module allows two different time discretizations of our
ODE system \eqref{eq:radiation_PDE}, \eqref{eq:cons_energy_correction}
and \eqref{eq:hydrogen_ionization}.  The first, described in
\cite{ReynoldsHayesPaschosNorman2009}, employs a two-level
$\theta$-method for the time discretization, and results in the
equations to compute the time-evolved solution $(E^n,e_c^n,\mn_{HI}^n)$,
\begin{align}
  \label{eq:radiation_PDE_theta}
  E^n - E^{n-1} &- \theta\dt\left(\nabla\cdot\(D\,\nabla E^n\) - \frac{\dot{a}}{a} E^n -
    c\kappa^n E^n + 4\pi\eta^n\right) \\ 
  \notag
  & - (1-\theta)\dt\left(\nabla\cdot\(D\,\nabla E^{n-1}\) - \frac{\dot{a}}{a} E^{n-1} -
    c\kappa^{n-1} E^{n-1} + 4\pi\eta^{n-1}\right) = 0, \\ 
  \label{eq:energy_correction_theta}
  e_c^n - e_c^{n-1} &- \theta\dt\left(-\frac{2\dot{a}}{a}e_c^{n} + G^{n} -
    \Lambda^{n}\right) - (1-\theta)\dt\left(-\frac{2\dot{a}}{a}e_c^{n-1} + G^{n-1} -
    \Lambda^{n-1}\right) = 0, \\
  \label{eq:hydrogen_theta}
  \mn_{HI}^n - \mn_{HI}^{n-1} &-
    \theta\dt\left(\alpha^{rec,n} \mn_e^{n} \mn_{HII}^{n} -
      \mn_{HI}^{n} \Gamma_{HI}^{ph,n}\right) -
    (1-\theta)\dt\left(\alpha^{rec,n-1} \mn_e^{n-1} \mn_{HII}^{n-1} -
      \mn_{HI}^{n-1} \Gamma_{HI}^{ph,n-1}\right) = 0,
\end{align}
where $0\le\theta\le 1$ defines the time-discretization, and where we
have assumed that the advective portions of \eqref{eq:radiation_PDE}
and \eqref{eq:hydrogen_ionization} have already been taken care of
through Enzo's hydrodynamics solver.  Recommended values of $\theta$
are 1 (backwards Euler) and $\frac12$ (trapezoidal,
a.k.a.~Crank-Nicolson).  Benefits of this approach include:
\begin{itemize}
\item Standard, easily understandable approach,
\item Unified treatment of all variables in implicit system,
\item Ease of analytical Jacobians for efficient solution of the
  resulting implicit systems.
\item Theoretical asymptotic accuracy of $O(\dt)$ for $\theta=1$ and
  $O(\dt^2)$ for $\theta=\frac12$. 
\end{itemize}
However, a key drawback of this approach is that it knows nothing
about the inherent constraints on the solution variables, i.e.~$E\ge
0$, $e>0$ and $\mn_{HI}\ge 0$.  As a result, in problems with
rapidly-varying chemical states in which $\mn_{HI}$ in a
newly-irradiated finite volume cell should change by multiple orders 
of magnitude in a single time step, these simple linear or quadratic
interpolations in time can result in negative solution values due to
time discretization error.  Therefore, in using the above system 
\eqref{eq:radiation_PDE_theta}-\eqref{eq:hydrogen_theta}, it is
imperative that the user allow {\em very conservative} time step sizes
to be used (see section \ref{sec:dt_selection}).

As a result, we have also implemented a second approach to time
discretization that should be much more robust, while attempting to
remain as accurate as possible.  This approach is based on a new
implicit-time version of the {\em quasi-steady-state approximation}.
In this approach, instead of approximating the solution to the exact
ODEs, we exactly solve approximate ODEs.  Here, if we assume in each
equation that all but the time-evolving variable are held constant
throughout the time step, we could consider the ODE system
\begin{align}
  \label{eq:energy_correction_qss}
  \partial_t e_c &= -\frac{2\dot{a}}{a}e_c + G(\overline{E},\overline{\mn_{HI}}) - \Lambda(\overline{E},e,\overline{\mn_{HI}}), \\
  \label{eq:hydrogen_qss}
  \partial_t \mn_{HI} &= \alpha^{rec}(\overline{e}) \mn_e \mn_{HII} - \mn_{HI} \Gamma_{HI}^{ph}(\overline{E}),
\end{align}
where $\overline{u}$ denotes a field $u$ that is assumed fixed
throughout a time step.  With this approximation,
\eqref{eq:energy_correction_qss}-\eqref{eq:hydrogen_qss} may be
written as
\begin{align}
  \label{eq:energy_correction_qss2}
  \partial_t e_c &= Q - Pe_c, \\
  \label{eq:hydrogen_qss2}
  \partial_t \mn_{HI} &= a \mn_{HI}^2 + b\mn_{HI} + c,
\end{align}
where $P$, $Q$, $a$, $b$ and $c$ are constant throughout the time
step.  These may be solved analytically to full accuracy for any time
step size $\dt$.  We write these analytical solvers as
\begin{align}
  \label{eq:energy_correction_qss3}
  e_c(t) &= \text{sol}_e\left(\overline{E},\overline{\mn_{HI}},e_c^{n-1},t\right), \\
  \label{eq:hydrogen_qss3}
  \mn_{HI}(t) &= \text{sol}_{HI}\left(\overline{E},\overline{e},\mn_{HI}^{n-1},t\right).
\end{align}
Since such an approach does not work well for PDEs (e.g.~equation
\eqref{eq:radiation_PDE}), and since the radiation equation constraint
is typically less problematic than the gas energy and chemistry, we
still use the $\theta$-method to discretize $E$.  Putting these
together, our second approach to time discretization uses the
equations 
\begin{align}
  \label{eq:radiation_PDE_iqss}
  E^n - E^{n-1} &- \theta\dt\left(\nabla\cdot\(D\,\nabla E^n\) - \frac{\dot{a}}{a} E^n -
    c\kappa^n E^n + 4\pi\eta^n\right) \\ 
  \notag
  & - (1-\theta)\dt\left(\nabla\cdot\(D\,\nabla E^{n-1}\) - \frac{\dot{a}}{a} E^{n-1} -
    c\kappa^{n-1} E^{n-1} + 4\pi\eta^{n-1}\right) = 0, \\ 
  \label{eq:energy_correction_iqss}
  e_c^n &- \text{sol}_e\left(\frac{E^{n-1}+E^n}{2},\frac{\mn_{HI}^{n-1}+\mn_{HI}^n}{2},e_c^{n-1},t\right) = 0, \\
  \label{eq:hydrogen_iqss}
  \mn_{HI}^n &- \text{sol}_{HI}\left(\frac{E^{n-1}+E^n}{2},\frac{e^{n-1}+e^n}{2},\mn_{HI}^{n-1},t\right) = 0.
\end{align}
Benefits of this approach (as opposed to the $\theta$ method) include:
\begin{itemize}
\item Robust solvers for gas energy and chemistry that can {\em never}
  result in negative solution values,
\item Fully implicit formulation in which all updated solution values
  $E^n$, $e_c^n$ and $\mn_{HI}^n$ depend on one another,
\item Asymptotic time accuracy of $O(\dt^2)$ at $\theta=\frac12$,
  and remarkably good accuracy at even large $\dt$.
\end{itemize}
Unfortunately, however, the use of these solvers makes analytical
derivation of Jacobians very difficult, so they must be approximated,
requiring additional computation per time step.


\subsection{Inexact Newton-Krylov-Multigrid solver}
\label{sec:nkmg}
No matter which time integration strategy is selected from section
\ref{sec:iqss},
\eqref{eq:radiation_PDE_theta}-\eqref{eq:hydrogen_theta} or 
\eqref{eq:radiation_PDE_iqss}-\eqref{eq:hydrogen_iqss}, we solve a
nonlinear system of equations 
\begin{align}
  \notag
  f_E(E,e_c,\mn_{HI}) &= 0, \\
  \label{eq:nonlinear_system}
  f_e(E,e_c,\mn_{HI}) &= 0, \\
  \notag
  f_{HI}(E,e_c,\mn_{HI}) &= 0,
\end{align}
at each time step to obtain the updated solution variables $E^n$,
$e_c^n$ and $\mn_{HI}^n$.  Grouping these together as $f(U)=0$, for
$U=(E,e_c,\mn_{HI})$, we solve the system \eqref{eq:nonlinear_system}
using a {\em globalized Inexact Newton's Method}
\cite{Kelley1995,KeyesReynoldsWoodward2006,KnollKeyes2004}:
\begin{quote}
Given an initial guess $U_0\approx U(t^n)$, we iterate toward a
solution $U^n$ satisfying $f(U^n)\approx 0$:
\begin{itemize}
\item[(a.)] Approximately solve the linear system $J_k S_k = -f_k$:
\begin{itemize}
  \item We find $S_k$ so that $\|J(U_k) S_k + f(U_k)\| <
    \delta_k$. 
  \item $\delta_k$ is a linear tolerance, chosen as $\delta_k =
    C\|f(U_k)\|$, where $C$ is a user-defined input.
  \item $J(U_k)$ is the Jacobian of $f$, $J(U_k) = \partial_U f(U_k)$
  \item We solve this system using a multigrid-preconditioned
    conjugate gradient iteration.
  \end{itemize}
\item[(b.)] Find a line-search parameter $\lambda_k$ so that
  \[
    \lambda_{min} \le \lambda \le \lambda_{max}
    \qquad\text{and}\qquad \|f(U_k + \lambda_k S_k)\| < \|f(U_k)\|.
  \] 
\item[(c.)] Update $U_{k+1} = U_k + \lambda_k S_k$.
\item[(d.)] Stop the iteration when
  \[
    \left(\frac1N \sum_{i=1}^N f(U_k)^2 \right)^{1/2} \ < \
    \varepsilon \ \ll \ 1.
  \]
\end{itemize}
\end{quote}



\subsection{Solver initial guess}
\label{sec:initial_guess}

Performance of the Newton iteration is highly influenced by a good
initial guess: a good guess results in a robust and efficient method,
while a poor guess may take much longer to converge (if at all).  We
therefore have a number of options that may be used to determine the
initial guess $U_0$:
\begin{itemize}
\item[0.] Use the previous solution 
  $U_0 = U^{n-1}$.
\item[1.] Use an explicit predictor: $U_0 = U^{n-1} + \dt
  \mathcal L(U^{n-1})$, where $\mathcal L$ comprises all of the
  spatially-local physics (i.e.~no diffusion).
\item[2.] Use an explicit predictor: $U_0 = U^{n-1} + \dt
  \mathcal R(U^{n-1})$, where $\mathcal R$ comprises all of the
  physics.
\item[3.] Use a partial explicit predictor: $U_0 = U^{n-1} + \frac{\dt}{10}
  \mathcal L(U^{n-1})$.
\item[4.] Use a partial explicit predictor: $U_0 = U^{n-1} + \frac{\dt}{10}
  \mathcal R(U^{n-1})$.
\item[5.] Use an analytic predictor of spatially-local physics: $U_0
  = (E^{n-1},\text{sol}_e,\text{sol}_{HI})$. 
\end{itemize}

\subsection{Time-step selection}
\label{sec:dt_selection}

Time steps are chosen adaptively in an attempt to control error in the
calculated solution.  To this end, we first define an heuristic
measure of the time accuracy error in a specific variable as
\begin{align}
\label{eq:time_error}
  e = \left(\frac1N \sum_{i=1}^N
    \left(\frac{u_i^{n}-u_i^{pred}}{\omega_i}\right)^p\right)^{1/p}, 
\end{align}
where the weighting vector $\omega$ is given by
\begin{align}
\label{eq:time_weighting}
  \omega_i &= \sqrt{u_i^n u_i^{pred}} + 10^{-3}, \quad i=1,\ldots,N, \\
  \omega_i &= |e_{c,i} + e_{h,i}| + 10^{-3}, \quad i=1,\ldots,N,
\end{align}
i.e.~we scale the radiation and chemistry change by the geometric mean
of the old and new states, and scale the gas energy change by the new
total gas energy, adding on a floor value of $10^{-3}$ in case either
of the states are too close to zero.  This approach works well when
the internal solution variables are unit-normalized, or at least close
to unit-normalized, since the difference between the predicted
solution $u^{pred}$ and the resulting solution $u^n$, divided by this
weighting factor $\omega$, should give a reasonable estimate of the
number of significant digits that are correct in the solution.

With these error estimates \eqref{eq:time_error} for each variable, we
set the new time step size based on the previous time step size and a
user-input tolerance $\tau_{\text{tol}}$ as
\begin{align}
\label{eq:time_estimate}
  \dt^{n} = \frac{\tau_{\text{tol}} \dt^{n-1}}{e}.
\end{align}
The final estimated time step size is given as the minimum of the
three,
\begin{align}
\label{eq:FLD_time_estimate}
  \dt_{\text{FLD}}^{n} =
  \min\{\dt_{E}^{n},\dt_{e_c}^{n},\dt_{HI}^{n}\}. 
\end{align}
The user may override this adaptive time step with the inputs
$\dt_{\text{max}}$ and $\dt_{\text{min}}$. 

We further note that when run in combination with Enzo's hydrodynamics
routines, both modules will limit their maximum time step sizes to the
minimum of $\dt_{\text{FLD}}$ and $\dt_{\text{CFL}}$, where
$\dt_{\text{CFL}}$ is the time step size that Enzo's other routines
would normally take.  As a result, in some physical regimes, the
global time step size will be limited based on the
radiation/ionization/heating time scale, and in other regimes it will
be limited by the hydrodynamic time scale.




\subsection{Variable rescaling}
\label{sec:variable_rescaling}

In case Enzo's standard unit non-dimensionalization with 
{\tt DensityUnits}, {\tt LengthUnits} and {\tt TimeUnits} is
insufficient to render the resulting solver values $E$, $e_c$ and
$n_{HI}$ to have nearly unit magnitude, the user may input additional
variable scaling factors to be used inside the {\tt gFLDProblem}
module.  Denoting the user-input values $s_E$, $s_e$ and $s_{\mn}$,
then all solvers within this module will rescale Enzo's internal
variables to 
\begin{align}
\label{eq:variable_rescaling}
  \tilde{E} = E / s_E, \qquad \tilde{e}_c = e_c / s_e, \qquad \tilde{\mn}_{HI} = \mn_{HI} / s_{\mn},
\end{align}
and use these rescaled values $\tilde{E}$, $\tilde{e}_c$ and
$\tilde{\mn}_{HI}$ instead of Enzo's values $E$, $e_c$ and
$\mn_{HI}$.  If the user does not know appropriate values for these
scaling factors {\em a-priori}, a generally-applicable rule of thumb
is to first run their simulation for a small number of time steps and
investigate Enzo's HDF5 output files to see the magnitude of values
stored internally by Enzo; if these are far from unit-magnitude, these 
scaling factors should be used.



\subsection{Boundary conditions}
\label{sec:boundary_conditions}

As the radiation equation \eqref{eq:radiation_PDE} is parabolic,
boundary conditions must be supplied on the radiation field $E$.  The
{\tt gFLDProblem} module allows three types of boundary conditions to
be placed on the radiation field:
\begin{itemize}
\item[0.] Periodic,
\item[1.] Dirichlet, i.e.~$E(x,t) = g(x), \; x\in\partial\Omega$, and
\item[2.] Neumann, i.e.~$\nabla E(x,t)\cdot n = g(x), \; x\in\partial\Omega$.
\end{itemize}
In most cases, the boundary condition types (and values of $g$) are
problem-dependent.  When adding new problem types, these conditions
should be set near the bottom of the file {\tt gFLDProblem\_Initialize.C}, 
otherwise these will default to either (a) periodic, or (b) will use
$g=0$ along with the user input boundary condition types.



\section{Concluding remarks}
\label{sec:conclusions}

We wish to remark that the module is not incredibly large (one header
file, 21 C++ files, 10 F90 files), and all files begin with the 
{\tt gFLDProblem} prefix.  While we have strived to ensure that the
module is bug-free, there is still work to be done in enabling
additional physics, including Helium chemistry and more advanced
time-stepping interactions with the rest of Enzo (especially when
ionization sources ``turn on'' abruptly).  

Feedback/suggestions are welcome.


\bibliography{sources}
\bibliographystyle{siam}
\end{document}

